\documentclass{jsarticle}
\AtBeginDvi{\special{papersize=91truemm,55truemm}}  % dviウェアにサイズを伝える
\renewcommand{\kanjifamilydefault}{\gtdefault}      % デフォルトをゴシックに
\usepackage[dvipdfm]{graphicx,pict2e}
\usepackage{calc}
\usepackage[dvipdfm,dvipsnames]{xcolor}
\usepackage[T1]{fontenc}
\usepackage[utf8]{inputenc}
\usepackage[sc]{mathpazo}
\usepackage[scaled]{beramono}
\usepackage[scaled=.8]{helvet}  % フォントの設定はご自由に
\usepackage[deluxe]{otf} % otf なければコメントアウト
\usepackage{okumacro}           % kintou, ruby に必要
\pagestyle{empty}
% マージン調整
\setlength{\hoffset}{0in}
\setlength{\voffset}{0in}
\setlength{\headheight}{0in}
\setlength{\headsep}{0in}
\setlength{\oddsidemargin}{-1truein}
\setlength{\topmargin}{-1truein}
\begin{document}
\setlength{\unitlength}{1truemm} %picture環境の単位が1mmになる
\begin{picture}(91,55)(0,0)
 \put(-4, 36){\includegraphics[clip, height=20truemm]{titech_logo}}
 \put(22, 24){\huge \kintou{7zw}{\mcfamily\bfseries 東工\hspace{1zw}花子}}
 \put(22, 32){{准教授\hspace{1zw}工学博士}}
 \put(36, 18){\resizebox{!}{.7zw}{\textcolor{SkyBlue}{国立大学法人} \textcolor{Blue}{\bfseries 東京工業大学}}}
 \put(36, 15){\resizebox{!}{.6zw}{総合理工学研究科}}
 \put(36, 12){\resizebox{!}{.6zw}{物質科学創造専攻}}
 \put(36,9.1){\ttfamily\resizebox{49mm}{!}{〒226-8503\hspace{.2zw}神奈川県横浜市緑区長津田町\hspace{.2zw}4259-S3-34}}
 \put(36,6.5){\ttfamily\resizebox{!}{.45zw}{Tel:\hspace{.2zw}045-924-XXXX Fax:\hspace{.2zw}045-924-XXXX}}
 \put(36,  4){\ttfamily\resizebox{!}{.45zw}{E-mail:\hspace{.2zw}xxxxxxxx@pe.titech.ac.jp}}
\end{picture}
\end{document}
